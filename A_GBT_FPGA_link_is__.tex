A GBT-FPGA link is used to emulate the GBTx serial link [3], and can also be instantiated on back-end FPGAs. Clocking resources are external to the GBT Bank, providing the user flexibility with connecting the different clocks. Depending upon user requirements, the GBT-FPGA core can be implemented using one or more GBT links in two different modes:
\begin{enumerate}
\item Standard 
\end{enumerate}
\begin{enumerate}
\item Latency-optimized (used when fixed, low and deterministic latency required for Tx, Rx, or both)
\end{enumerate}
In addition, the GBTx offers two different encoding schemes (Figure \ref{FIG_ID}):
\begin{enumerate}
\item GBT-Frame mode (based on Reed-Solomon encoding)
\end{enumerate}
\begin{enumerate}
\item Wide-Bus mode (no encoding)
\end{enumerate}
Both schemes use a \textbf{120-bit} bus to/from a GBT-FPGA link operating at 4.8Gbps
